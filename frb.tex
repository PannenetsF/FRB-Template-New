\documentclass[lang=cn,12pt]{frbpaper}


\title{本科毕设开题报告}
\subtitle{毕设题目}
\author{你的大名}
\stucode{23333333}
\stumajor{摸鱼科学与工程}
\stusupervisor{导师大名}
\stuschool{摸鱼学院}


\headofpaper{北京航空航天大学集成电路科学与工程学院}


\begin{document}

\maketitle

% Please add the following required packages to your document preamble:
% \usepackage{multirow}
% \usepackage{graphicx}
% Please add the following required packages to your document preamble:
% \usepackage{multirow}
% \usepackage{graphicx}
\begin{table}[htp]\sihao
    \centering
    \label{tab:my-table}
    
    \begin{tabular}{|cccccccccc|}
    \hline
    \multicolumn{1}{|c|}{学生姓名} & \multicolumn{3}{c|}{摸鱼怪} & \multicolumn{1}{c|}{学号} & \multicolumn{2}{c|}{23333333} & \multicolumn{2}{c|}{专业} & 摸鱼科学与工程 \\ \hline
    \multicolumn{1}{|c|}{校内导师} & \multicolumn{9}{c|}{摸鱼圣者} \\ \hline
    \multicolumn{1}{|c|}{\multirow{4}{*}{校外导师}} & \multicolumn{2}{c|}{姓名} & \multicolumn{7}{c|}{单位} \\ \cline{2-10} 
    \multicolumn{1}{|c|}{} & \multicolumn{2}{c|}{域外圣者} & \multicolumn{7}{c|}{域外} \\ \cline{2-10} 
    \multicolumn{1}{|c|}{} & \multicolumn{2}{c|}{职称} & \multicolumn{3}{c|}{职务} & \multicolumn{2}{c|}{电话} & \multicolumn{2}{c|}{邮箱} \\ \cline{2-10} 
    \multicolumn{1}{|c|}{} & \multicolumn{2}{c|}{圣者} & \multicolumn{3}{c|}{摸鱼} & \multicolumn{2}{c|}{23336666} & \multicolumn{2}{c|}{\href{mailto:me@somewhere.com}{me@somewhere.com}} \\ \hline
    \multicolumn{1}{|c|}{毕设地点} & \multicolumn{9}{c|}{?} \\ \hline
    \multicolumn{2}{|c|}{毕设题目} & \multicolumn{8}{c|}{?} \\ \hline
    \multicolumn{2}{|c|}{论文类型} & \multicolumn{8}{c|}{工程设计(\checkmark) 科学研究() 实验() 其他()} \\ \hline
    \multicolumn{10}{|l|}{\begin{tabular}[l]{@{}l@{}}
        工作内容:\\ 
        \begin{tabular}{p{\textwidth}}
            工作内容是指劳动者具体从事什么种类或内容的劳动,是劳动合同确定劳动者应当履行劳动义务的主要内容,包括劳动者从事劳动的工种、岗位、工作范围、工作任务、工作职责、劳动定额、质量标准等。工作内容条款是劳动合同的核心条款之一。
            它是用人单位聘用劳动者的目的,也是劳动者取得劳动报酬的缘由。该条款的约定应当明确具体,便于劳动者判断自己是否胜任该工作,是否愿意从事该工作,也便于双方遵照执行。
        \end{tabular} \\
        预期目标及设计技术要求:(明确且可考核) \\
        \begin{tabular}{p{\textwidth}}
            工作内容是指劳动者具体从事什么种类或内容的劳动,是劳动合同确定劳动者应当履行劳动义务的主要内容,包括劳动者从事劳动的工种、岗位、工作范围、工作任务、工作职责、劳动定额、质量标准等。工作内容条款是劳动合同的核心条款之一。
            它是用人单位聘用劳动者的目的,也是劳动者取得劳动报酬的缘由。该条款的约定应当明确具体,便于劳动者判断自己是否胜任该工作,是否愿意从事该工作,也便于双方遵照执行。
        \end{tabular} \\
        \begin{minipage}{.9\linewidth}
            \begin{flushright}
                指导教师签字:\;\;\;\;\;\;\;\;\;\; \\ \vspace{0.5cm}
                \hspace{1cm}年\hspace{1cm}月\hspace{1cm}日
            \end{flushright} 
        \end{minipage}
        \end{tabular}} \\ \hline
    \end{tabular}%
    \end{table}

\makemain

% \maketask


\section{课题来源、研究目的及意义}
\section{国内外研究现状}
\section{研究内容}
\section{研究方案及技术路线}
\section{进度安排}
\section{预期目标}
\section{参考文献}
% \addcontentsline{toc}{section}{参考文献}

\bibliography{frbrefer}


\appendix
\addappheadtotoc

\end{document}